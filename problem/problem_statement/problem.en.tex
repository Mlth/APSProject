\problemname{Solitaris}
Greg is an intergalactic adventurer and biologist who is always on the hunt to discover new species. He is currently on the planet of Solitaris, which is said to be inhabited by a mythical creature. Greg has heard that the creature is most likely to reside in low-altitude areas, whether it be on land or in water.
\\\\
Greg has noted all the places where the mythical creature has allegedly been spotted by other travelers. More importantly, he has noted the altitude of all these places. Greg regularly goes on expeditions in different areas of Solitaris. To make his expeditions less time-consuming, Greg plans to prepare each expedition by finding the point with the lowest altitude in the given area, since this is where the creature is most likely to be found. You are also interested in the mythical creature and have offered to help Greg in his search. Greg therefore sends you a list of the altitude of all the points where the creature has been spotted, and agrees with you that every time he starts a new expedition, he sends you an interval between the first and last point he can search in the area he is going to, and your task is to find the lowest altitude so Greg knows where to search.
\\\\
This task is slightly complicated by the fact that Solitaris’s terrain is ever-changing. When Greg landed on the planet, it was completely flat, meaning that all points of interest in the list you received from Greg have the same altitude. However, frequent earthquakes cause the altitude of different points on the planet to change. Greg has promised to send you a message every time an earthquake changes the altitude of any of the points he has given you. This way, you know which points have either increased or decreased in altitude.
\section*{Input}
The first line of input consists of three integers. Firstly, the number of points of interest, 1 $\leq$ \emph{N} $\leq$ 100 000. Secondly, the amount of earthquakes and/or expeditions, 1 $\leq$ \emph{O} $\leq$ 125 000. Thirdly, the initial altitude of all points of interest, -10 000 $\leq$ \emph{S} $\leq$ 10 000. The first line of input is followed by \emph{O} lines each consisting of an operation, either “quake” or "expedition":
\begin{itemize}
    \item A “quake” is followed by 2 integers \emph{i} and \emph{v}. 1 $\leq$ \emph{i} $\leq$ \emph{N} is the point that has changed in altitude and -10 000 $\leq$ \emph{v} $\leq$ 10 000 is the new altitude.
    \item An “expedition” is followed by 2 integers, \emph{s} and \emph{e}, which represent the start and end point of interest in the expedition. The following can be assumed about the size of \emph{s} and \emph{e}: 1 $\leq$ \emph{s} $\leq$ \emph{e} $\leq$ \emph{N}.
\end{itemize}
\section*{Output}
Output for each expedition the lowest altitude in the given interval, i.e. the altitude of the point where the creature is most likely to be found in that area.\\